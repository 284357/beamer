\documentclass{beamer}
\usetheme {default}
\title{Cwiczenia}
\subtitle{Opis cwiczen na poszczegolne partie miesniowe}
\author{Jakub Felczynski}
\date{\today}
\begin{document}
\begin{frame}{Cwiczenia}
\titlepage
\end{frame}
\section{CWICZENIA NA MIESNIE KLATKI PIERSIOWEJ:}
\begin{frame}{CWICZENIA NA MIESNIE KLATKI PIERSIOWEJ:}
1. Wyciskanie sztangi na lawce poziomej
Wykonanie:
Poloz sie na plecach na lawce poziomej, stopy na podlodze.
Chwyc sztange na szerokosc nieco wieksza niz barki.
Opusc sztange do klatki piersiowej, lokcie powinny byc zgiete pod katem okolo 75-90 stopni.
Wyciskaj sztange do gory, prostujac ramiona, nie blokujac lokci na gorze
2. Pompki
Wykonanie:
Ustaw sie w pozycji plank, rece na szerokosc barkow.
Cialo w linii prostej od glowy do piet.
Zginaj lokcie, opuszczajac cialo w kierunku podlogi, nie odchylajac bioder.
Kiedy klatka piersiowa prawie dotknie podlogi, wypchnij sie do gory.
3. Wyciskanie hantli na lawce skosnej
Wykonanie:
Usiadz na lawce skosnej z hantlami w dloniach.
Hantle trzymaj na wysokosci klatki piersiowej, lokcie zgiete.
Wyciskaj hantle do gory, prostujac ramiona.
Opusc je z powrotem do poziomu klatki piersiowej.
4. Rozpietki z hantlami
Wykonanie:
Poloz sie na plecach na lawce poziomej z hantlami w rekach.
Trzymaj hantle nad klatka piersiowa, ramiona lekko zgiete.
Powoli opuszczaj hantle na boki, czujac rozciaganie w klatce piersiowej.
Zatrzymaj sie, gdy rece beda rownolegle do podlogi, a nastepnie wroc do pozycji wyjsciowej.
\end{frame}
\section{CWICZENIA NA MIESNIE PLECOW:}
\begin{frame}{CWICZENIA NA MIESNIE PLECOW:}
1. Martwy ciag
Wykonanie:
Stan z nogami na szerokosc bioder, sztanga przed stopami.
Pochyl sie w biodrach, utrzymujac prosta plecy, i chwyc sztange na szerokosc nieco wieksza niz barki.
Zacisnij brzuch, a nastepnie wyprostuj nogi i biodra, unoszac sztange w gore.
Na szczycie ruchu, napnij miesnie posladkow, a nastepnie powoli opusc sztange na podloge.
2. Wioslowanie hantlami
Wykonanie:
Stan z nogami na szerokosc bioder, hantle w rekach.
Pochyl sie w biodrach, trzymajac plecy prosto.
Zegnij lokcie i przyciagnij hantle do klatki piersiowej.
Powoli opusc hantle do pozycji wyjsciowej.
3. Przyciaganie drazka do klatki piersiowej
Wykonanie:
Usiadz na maszynie do przyciagania drazka, stopy na podlodze.
Chwyc drazek na szerokosc barkow, wyprostuj ramiona.
Przyciagnij drazek do klatki piersiowej, zginajac lokcie.
Powoli wroc do pozycji wyjsciowej.
4. Odwrotne rozpietki (Reverse Fly)
Wykonanie:
Usiadz na lawce pochylonej do przodu, hantle w dloniach.
Trzymaj hantle z ramionami lekko zgietymi.
Unikaj zbytniego zginania nadgarstkow i ramion.
Opusc hantle na boki, czujac rozciaganie w plecach, a nastepnie wroc do pozycji wyjsciowej.
\end{frame}
\section{CWICZENIA NA BICEPS:}
\begin{frame}{CWICZENIA NA BICEPS:}
1. Wyciskanie sztangi stojac (Bicep Curl)
Wykonanie:
Stan prosto, stopy na szerokosc bioder, trzymajac sztange w dloniach, rece opuszczone wzdluz ciala.
Chwyc sztange na szerokosc barkow, dlonie skierowane do przodu.
Zginaj lokcie, podnoszac sztange w kierunku klatki piersiowej, nie ruszajac gornej czesci ciala.
Powoli opusc sztange do pozycji wyjsciowej.
2. Wyciskanie hantli siedzac
Wykonanie:
Usiadz na lawce z prostym oparciem, trzymajac hantle w dloniach, rece opuszczone wzdluz ciala.
Chwyc hantle tak, aby dlonie byly skierowane do siebie.
Zginaj lokcie, unoszac hantle do ramion, a nastepnie wroc do pozycji wyjsciowej.
3. Uginanie ramion na modlitewniku (Preacher Curl)
Wykonanie:
Usiadz na modlitewniku, trzymajac sztange lub hantle w dloniach.
Oprzyj ramiona na oparciu, rece opuszczone.
Zginaj lokcie, unoszac ciezar w kierunku ramion.
Powoli opusc ciezar do pozycji wyjsciowej.
4. Uginanie ramion z supinacja (Hammer Curl)
Wykonanie:
Stan prosto, trzymajac hantle w dloniach, rece opuszczone wzdluz ciala.
Upewnij sie, ze dlonie sa skierowane do siebie (neutralny chwyt).
Zginaj lokcie, unoszac hantle w kierunku ramion, az do pelnego zgiecia.
Powoli opusc hantle do pozycji wyjsciowej.
\end{frame}
\section{CWICZENIA NA TRICEPS:}
\begin{frame}{CWICZENIA NA TRICEPS:}
1. Wyciskanie sztangi lezac (Close-Grip Bench Press)
Wykonanie:
Poloz sie na plecach na lawce poziomej, stopy na podlodze.
Chwyc sztange na szerokosc ramion (blizej ciala niz w klasycznym wyciskaniu).
Opusc sztange w kierunku klatki piersiowej, trzymajac lokcie blisko ciala.
Wyciskaj sztange w gore, prostujac ramiona.
2. Pompki na wasko
Wykonanie:
Ustaw sie w pozycji pompki, dlonie blizej siebie (na szerokosc ramion lub nieco wezej).
Cialo w linii prostej od glowy do piet.
Zginaj lokcie, opuszczajac cialo w kierunku podlogi.
Wypchnij sie do gory, wracajac do pozycji wyjsciowej.
3. Francuskie wyciskanie hantli (French Press)
Wykonanie:
Usiadz lub poloz sie na lawce, trzymajac hantle w obu dloniach nad glowa (ramiona wyprostowane).
Zginaj lokcie, opuszczajac hantle za glowe, a nastepnie wroc do pozycji wyjsciowej.
Utrzymuj lokcie blisko glowy przez caly czas.
4. Dipy na poreczach
Wykonanie:
Stan miedzy poreczami, chwyc je obiema rekami.
Unies cialo, prostujac ramiona.
Zginaj lokcie, opuszczajac cialo w dol, a nastepnie wroc do pozycji wyjsciowej.
Utrzymuj cialo blisko poreczy.
\end{frame}
\section{CWICZENIA NA MIESNIE BARKOW:}
\begin{frame}{CWICZENIA NA MIESNIE BARKOW:}
1. Wyciskanie sztangi nad glowe
Jak wykonac:
Stan w pozycji wyprostowanej, trzymajac sztange na wysokosci klatki piersiowej.
Upewnij sie, ze twoje stopy sa na szerokosc bioder.
Wciagnij brzuch i wypchnij sztange w gore, az ramiona beda wyprostowane.
Powoli opusc sztange z powrotem do klatki piersiowej.
2. Unoszenie hantli bokiem
Jak wykonac:
Stan w pozycji wyprostowanej, trzymajac hantle w obu rekach przy bokach.
Z lekko ugietymi lokciami, unoszac hantle w bok, az osiagna wysokosc barkow.
Powoli opusc je z powrotem do pozycji startowej.
3. Wyciskanie hantli siedzac
Jak wykonac:
Usiadz na lawce z oparciem, trzymajac hantle na wysokosci barkow.
Upewnij sie, ze stopy sa stabilnie na ziemi.
Wyciskaj hantle w gore, az ramiona beda w pelni wyprostowane.
Powoli opusc hantle do poziomu barkow.
4. Arnold Press
Jak wykonac:
Usiadz lub stan, trzymajac hantle na wysokosci klatki piersiowej, dlonmi skierowanymi do ciala.
Wyciskaj hantle do gory, obracajac dlonie tak, aby w koncowej pozycji byly skierowane na zewnatrz.
Nastepnie, obracajac dlonie, opusc hantle z powrotem do pozycji startowej.
\end{frame}
\section{CWICZENIA NA MIESNIE BRZUCHA:}
\begin{frame}{CWICZENIA NA MIESNIE BRZUCHA:}
1. Plank
Jak wykonac:
Poloz sie na brzuchu, opierajac sie na przedramionach i palcach stop.
Utrzymaj cialo w linii prostej od glowy do piet, napinajac miesnie brzucha i posladkow.
Wytrzymaj w tej pozycji 30-60 sekund.
Staraj sie oddychac rownomiernie.
2. Crunches (brzuszki)
Jak wykonac:
Poloz sie na plecach, ugnij kolana i stopy trzymaj na ziemi.
Rece moga byc zalozone za glowa lub skrzyzowane na klatce piersiowej.
Unies gorna czesc ciala, napinajac brzuch, a nastepnie powoli opusc sie do pozycji wyjsciowej.
3. Russian Twists
Jak wykonac:
Usiadz na podlodze, ugnij kolana i unies stopy nad ziemie, rownoczesnie pochylajac sie lekko do tylu.
Trzymajac rece razem lub z hantlem, obracaj tulow na boki, dotykajac podlogi obok bioder.
4. Unoszenie nog w lezeniu
Jak wykonac:
Poloz sie na plecach z rekami wzdluz ciala.
Unies nogi prosto w gore, utrzymujac je razem.
Powoli opuszczaj nogi w dol, az beda tuz nad ziemia, a nastepnie wroc do pozycji wyjsciowej.
\end{frame}
\section{CWICZENIA NA MIESNIE NOG:}
\begin{frame}{CWICZENIA NA MIESNIE NOG:}
1. Przysiady (Squats)
Jak wykonac:
Stan w pozycji wyprostowanej, stopy na szerokosc bioder.
Napnij brzuch i wypchnij biodra do tylu, jakbys siadal na niewidzialnym krzesle.
Zgin kolana, az uda beda rownolegle do podlogi, pamietajac, aby kolana nie wychodzily poza palce stop.
Wroc do pozycji wyjsciowej, prostujac nogi.
Wykonaj 10-15 powtorzen.
2. Wykroki (Lunges)
Jak wykonac:
Stan w pozycji wyprostowanej, a nastepnie zrob krok do przodu jedna noga.
Zgin oba kolana, az tylne kolano bedzie blisko podlogi (nie dotykaj).
Upewnij sie, ze przednie kolano nie wychodzi poza palce stop.
Odepchnij sie od przedniej nogi i wroc do pozycji wyjsciowej.
3. Martwy ciag na prostych nogach (Deadlifts)
Jak wykonac:
Stan z stopami na szerokosc bioder, trzymajac sztange lub hantle przed soba.
Utrzymujac proste plecy, zegnij sie w biodrach, opuszczajac ciezar wzdluz nog.
Utrzymaj lekkie ugiecie w kolanach.
Wroc do pozycji wyjsciowej, prostujac biodra i plecy.
\includegraphics[width=4cm,angle=45]{/home/jakub/Pobrane/obraz}
\end{frame}
\end{document}
